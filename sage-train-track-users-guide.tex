\documentclass[10pt,a4paper]{article}

\usepackage{amsmath}

\newcommand{\Out}{\text{Out}}
\newcommand{\Aut}{\text{Aut}}
\newcommand{\FN}{F_N}

\title{Free group automorphisms and train-tracks with Sage\\
User's Guide}
\author{Thierry Coulbois}
\date{\today}

\makeindex

\begin{document}


\maketitle



\section{Introduction}

The Train-track package was first written by Thierry Coulbois and
received contributions by Matt Clay and others.

It is primarily intended to implement the computation of a train-track
representative for automorphisms of free groups as introduced by
Mladen Bestvina and Mark Feighn~\cite{bf-train-track}.


Sage is based on Python. This is an object oriented language: the
\textbf{postfix convention} is used. For instance
\texttt{phi.train-track()} applies the method \texttt{train-track()}
to the object \texttt{phi}. Note that \index{method} is the object
oriented linguo for what mathematicians call ``function''.

You can always ask for \textbf{automatic completion} and \textbf{help} by using the TAB key: 
\begin{enumerate}
\item Hitting the TAB key after a letter offers all possible
  completions known to sage.
\item Hitting the TAB key after a dot shows all methods that can be
  applied to that object.
\item Hitting the TAB key after an opening parenthesis gives help on
  how this method should be used.
\end{enumerate}

Most methods have \textbf{verbose options} to display intermediate
computations. They are turned off by default, but you can supply a
\texttt{verbose=True} option or any non-negative number to get extra
details.

The main documentation for using this package is inline help and
automatically created documentation. This User guide is only
intended for beginners and general structure.

\section{Installation and files}

To use this package you first need a recent ($\geq 6.3$) distribution
of Sage. Then you need to download the files at

\section{Free groups and automorphisms}

\subsection{Creating free groups}

Probably you first need to create a \index{free group}. It can be specified by
its rank or a list of letters. You can also first create the \index{alphabet}.
\begin{verbatim}
sage: F=FreeGroup(3); F
   Free group over ['a', 'b', 'c']
sage: F=FreeGroup(['x0','x1','x3','x4']); F
   Free group over ['x0', 'x1', 'x3', 'x4']
sage: A=AlphabetWithInverses(5,type='a0')
sage: F=F(A); F
   Free group over ['a0', 'a1', 'a2', 'a3', 'a4']
\end{verbatim}
You can declare anything to be a letter, but beware that if letters are
not single ascii characters (like 'x0'), you will need to be careful
while going from Strings to Words.

\subsection{Free group elements}

Free group elements are words. They are created by
\begin{verbatim}
sage: F=FreeGroup(3)
sage: F('abA')
   word: abA
sage: w=F('abAaab'); w
   word: abAaab
sage: F.reduce(w)
   word: abab
\end{verbatim} 
Note that they are not reduced by default. 

Words can be multiplied and inverted easily:

\begin{verbatim}
sage: w=F('abA')
sage: w*w
   word: abbA
sage: w.inverse()
   word: aBA
sage: w**5
   word: abbbbbA
\end{verbatim}

Warning: be careful when the free group alphabet is not made of ascii letters:

\begin{verbatim}
sage: A=AlphabetWithInverses(3,type='x0')
sage: F=FreeGroup(A)
sage: ws='x0X0x1'
sage: w=F(ws); w
word: 'x0X0x1'
sage: F.reduce(w)
KeyError: 'x'
sage: w=F(['x0','X0','x1']); w
word: x0,X0,x1
sage: F.reduce(w)
word: x1
\end{verbatim}

\subsection{Free group automorphisms}

The parsing of free group automorphisms relies on that of
substitutions. Most of what you might expect should correctly create a
\index{free group automorphism}:
\begin{verbatim}
sage: phi=FreeGroupMorphism('a->ab,b->a'); phi
Automorphism of the Free group over ['a', 'b']:
a->ab,b->a
\end{verbatim}
Automorphisms can be composed, inverted (note that there is not test
of invertibility upon creation), exponentiated, applied to free group elements.
\begin{verbatim}
sage: phi=FreeGroupAutomorphism('a->ab,b->ac,c->a')
sage: phi=FreeGroupAutomorphism('a->c,b->ba,c->bcc')
sage: print phi*psi
a->a,b->acab,c->acaa
sage: print phi.inverse()
a->c,b->Ca,c->Cb
sage: print phi**3
a->abacaba,b->abacab,c->abac
sage: phi('aBc')
word: abC
\end{verbatim}

Note that there is a list of pre-defined automorphisms of free groups taken from the litterature:
\begin{verbatim}
sage: print free_group_automorphisms.Handel_Mosher_inverse_with_same_lambda()
a->b,b->c,c->Ba
\end{verbatim}

Also Free group automorphisms can be obtained as composition of
\index{elementary Nielsen automorphisms} (of the form $a\mapsto
ab$). Up to now they are rather called \index{Dehn twist}.
\begin{verbatim}
sage: F=FreeGroup(3)
sage: print F.dehnt_twist('a','c')
a->ac, b->b, c->c
sage: print F.dehn_twist('A','c')
a->Ca,b->b,c->c
sage: print F.dehn_twist('a','b',on_left=True)
a->ba,b->b,c->c
\end{verbatim}

If the free group as even rank $N=2g$, then it is the fundamental
group of an oriented surface ofgenus $g$ with one boundary
component. In this case the mapping class group of $S_{g,1}$ is a
subgroup of the outer automorphism group of $F_N$ and it is generated
by a collection of $3g-1$ Dehn twists along curves. Those \index{Dehn twist}s are accessed through:
\begin{verbatim}
sage: F=FreeGroup(4)
sage: print F.surface_dehn_twist(2)
a->a,b->ab,c->acA,d->adA
\end{verbatim}

Similarly the \index{braid} group $B_N$ is a subgroup of $\Aut(\FN)$ and its usual generators are obtained by:
\begin{verbatim}
sage: F=FreeGroup(3)
sage: print F.braid_automorphism(0)
a->c,b->b,c->caC
\end{verbatim}

Finally for statistical purpose, one can access \index{random
  automorphisms} or \index{random mapping classes} or \index{random
  braids}. The random elements are obtained by composition of a given number
of randomly chosen generators of these groups.
\begin{verbatim}
sage: F=FreeGroup(4)
sage: F.random_automorphism(8)
\end{verbatim}

\section{Graphs and maps}

Graphs and maps are used to represent free group automorphisms. A
graph here is a GraphWithInverses: it has a set of vertices and a set
of edges in one-to-one correspondance with the letters of an
AlphabetWithInverses: each non-oriented edge is a pair $\{e,\bar e\}$
of a letter of the alphabet and its inverse. This is complient with
Serre's view. As the alphabet has a set of positive letters there is a
default choice of orientation for edges.

The easiest graph is the \index{rose}:
\begin{verbatim}
sage: A=AlphabetWithInverses(3)
sage: G=GraphWithInverses.rose_graph(A)
sage: print G
Graph with inverses: a: 0->0, b: 0->0, c: 0->0
sage: G.plot()
\end{verbatim}
Otherwise a graph can be given by a variety of inputs like a list of
edges, etc. Graphs can easily be \index{plot}ted. Note that
\texttt{plot()} tries to lower the number of accidental crossing of
edges, using some thermodynamics and randomness, thus two calls of
\texttt{plot()} may output two different figures.

A number of operations on graphs are defined: subdividing, folding,
collapsing edges, etc.

Graphs come with maps between them: a map is a continuous map from a
graph to another which maps vertices to vertices and edges to
edge-paths. Again they can be given by a variety of means. As Graph
maps are intended to represent free group automorphisms a simple way
to create a graph map is from a free group automorphism:
\begin{verbatim}
sage: phi=free_group_automorphisms.tribonacci()
sage: print phi.rose_representative()
Topological representative:
Marked graph: a: 0->0, b: 0->0, c: 0->0
Marking: a->a, b->b, c->c
Edge map: a->ab, b->ac, c->a
\end{verbatim}
Remark that by default the rose graph is \textbf{marked}: it comes
with a marking from the rose (itself, but you should think of that one
as fixed) to the graph. Here the graph map is a graph self map as the
source and the target are the same.

Graph maps can also be folded, subdivided, etc. If the graphs are
marked then those operations will carry on the marking.

\section{Train-tracks}

The main feature and the main achievement of the program is to compute
train-track representative for (outer) automorphisms of free groups.



\end{document}
